{\parskip 1ex

%%%%%%%%%%%%%%%%%%%%%%%%%%%%%%%%%%%%%%%%%%%%%%%%%%%%

{\huge {\bf Workshops}}\\[5mm]


\begin{tabular}{lp{13.8cm}}
\hline
\multicolumn{2}{|c|}{{\bf Monday, June 27, 2011}}\\
\hline\\
%8:30 & {\bf Registration}	\\[2mm]
9:00-10:30 & {\bf Session 1. } \\[2mm]%Chair: -} \\[2mm]
10:30-10:45 & {\bf Break}\\[2mm]
10:45-12:15 & {\bf Session 2. } \\[2mm]%Chair: -} \\[2mm]
12:15-1:30 & {\bf Lunch}\\[2mm]
1:30-3:00 & {\bf Session 3. } \\[2mm]%Chair: -} \\[2mm]
3:00-3:30 & {\bf Break}\\[2mm]
3:30-5:00 & {\bf Session 4. } \\[2mm]%Chair: -} \\[2mm]
\end{tabular}


\begin{tabular}{lp{13.8cm}}
\hline
\multicolumn{2}{|c|}{{\bf Friday, July 1, 2011}}\\
\hline\\
9:00-10:30 & {\bf Session 1. } \\[2mm]%Chair: -} \\[2mm]
10:30-10:45 & {\bf Break}\\[2mm]
10:45-12:15 & {\bf Session 2. } \\[2mm]%Chair: -} \\[2mm]
12:15-1:30 & {\bf Lunch}\\[2mm]
1:30-3:00 & {\bf Session 3. } \\[2mm]%Chair: -} \\[2mm]
3:00-3:30 & {\bf Break}\\[2mm]
3:30-5:00 & {\bf Session 4. } \\[2mm]%Chair: -} \\[2mm]
\end{tabular}

\clearpage

%%%%%%%%%%%%%%%%%%%%%%%%%%%%%%%%%%%%%%%%%%%%%%%%%%%%
%======================================================================
{\Large {\bf  WS1.  RGB-D: Advanced Reasoning with Depth Cameras}}\\[5mm]
%======================================================================

{\bf  Schedule and Location:}

Full-day workshop, Monday, June 27, 2011\\
%9:30 am -  6:00 pm\\
Location: GFS 101 \\[4mm]
%Room M6 (1$^{st}$ floor), School of Economics and Business Administration, Universidad de Zaragoza, Spain\\[4mm]

{\bf  Organizers:}

Xiaofeng Ren Intel Labs Seattle\\
Dieter Fox, University of Washington\\
Jana Kosecka, George Mason University\\
Kurt Konolige, Willow Garage\\[4mm]

{\bf Description: }

The arrival of Microsoft Kinect, with \$150 a unit and 8 million sales in two months, is causing a revolution across robotics research landscapes. Affordable RGB-D cameras, with real-time synchronized color and dense depth, are to dramatically improve and fundamentally change robots. capabilities to perceive and interact with people and environments. Last year.s RGB-D workshop at RSS successfully brought together experts from multiple research fields with converging interests and discussed major RGB-D opportunities and challenges in robotics. This year, the main purpose of this workshop is to understand the scope and impact of the rapidly growing RGB-D-based research activities, to solicit and showcase in-progress RGB-D-based systems and applications, to clarify a research agenda for depth camera perception, and to coordinate efforts across communities to lead the emerging RGB-D revolution.



%======================================================================
\clearpage {\Large {\bf WS2. The State of Imitation Learning: Understanding its Applications and Promoting its Adoption}}\\[5mm]
%======================================================================

{\bf  Schedule and Location:}

Full-day workshop, Monday, June 27, 2011\\
%9:30 am - 6:00 pm\\
Location: GFS 201 \\[4mm]
%Room M1 (2$^{nd}$ floor), School of Economics and Business Administration, Universidad de Zaragoza, Spain\\[4mm]

{\bf  Organizers:}

Brenna Argall, École Polytechnique Fédérale de Lausanne\\
Nathan Ratliff, Google\\
David Silver, Carnegie Mellon University\\[4mm]


{\bf Description: }

Imitation learning has grown into a large field with applications across robotics, neural computation, and artificial intelligence. As the field has developed, ideas have sprouted from a wide range of motivations and applications resulting differing terminology and significant overlap; terms such as apprenticeship learning, learning from demonstration, inverse optimal control, and inverse reinforcement learning mean the same thing to some, while to others they have vastly different connotations. Imitation learning is already creating a stir within the robotics community as an effective and practical way to transfer our intuition to real world robotics systems, and has the potential to revolutionize the way we approach system development. In this workshop, we will examine the collection of subfields within imitation learning together and attempt to construct a formal taxonomy of the tools and techniques available to solidify its foundation and promote wider adoption with the robotics community.



%======================================================================
\clearpage {\Large {\bf WS3. Toward High-Performance Computing Support for the Analysis, Simulation, and lanning of Robotics Contact Tasks}}\\[5mm]
%======================================================================

{\bf  Schedule and Location:}

1.5 day workshop, Monday, June 27, 2011 and Tuesday, June 28, 2011 (morning)\\
%9:30 am - 6:00 pm\\
Location: GFS 202 (Monday) and GFS 101 (Tuesday morning) \\[4mm]
%Room M3 (1$^{st}$ floor), School of Economics and Business Administration, Universidad de Zaragoza, Spain\\[4mm]

{\bf  Organizers:}

Chris Carothers, Rensselaer Polytechnic Institute\\
Dan Negrut, University of Wisconsin\\
Jeff Trinkle,   Rensselaer Polytechnic Institute\\[4mm]

{\bf Description: }

A key need in robotics is reliable prediction of actions involving intermittent contact such as legged locomotion over rough terrain, grasp acquisition, multi-robot manipulation, and assembly. As the reliability and speed of prediction increase, more capable planners can be developed that produce robust plans for more complex tasks. Current software that can analyze and simulate contact tasks vary greatly in their speed, physical fidelity, and stability. The main goals of the proposed workshop are two-fold. First, the participants will assess the state of the art in the analysis, simulation, and planning of robotic contact tasks. Second, they will assess the needs of the robotics community for a software platform related to robotic contact tasks. If the recommendation of the workshop participants is to move ahead, then the organizers will seek sources of support to pursue development.



%======================================================================
\clearpage {\Large {\bf WS4.  Tutorial on Designing Social Behavior (CANCELED)}}\\[5mm]
%======================================================================



%======================================================================
\clearpage {\Large {\bf WS5.  ALONE - Autonomous Long-Term Operation in Novel Environments}}\\[5mm]
%======================================================================

{\bf  Schedule and Location:}

Full-day workshop, Monday, June 27, 2011\\
%9:30 am - 6:00 pm\\
Location: GFS 104 \\[4mm]
%Seminar M1 (lower floor), School of Economics and Business Administration, Universidad de Zaragoza, Spain.\\[4mm]

{\bf  Organizers:}

Jonathan Kelly, University of Southern California\\
Paul Newman, University of Oxford\\
Sebastian Thrun, Stanford University / Google\\[4mm]


{\bf Description: }

The majority of future robots will operate outside of the laboratory or the factory, performing tasks in challenging, dynamic environments. In many situations, removal from service for maintenance or adjustment may be infeasible or impossible. As such, these robots will need to function reliably and autonomously on time scales ranging from days to years. Developing adaptive and flexible systems capable of this level of independence will require significant technical advances. The aim of the ALONE workshop is to bring together researchers from a diverse range of sub-disciplines, to identify and discuss fundamental challenges related to long-term autonomous operation; focus will be given to robots working in particularly demanding locations, e.g., busy homes and offices, on existing road networks, underground, underwater, in space, and on remote planetary surfaces. The workshop will include short talks and panel discussions on topics including, but not limited to: resource-constrained long-horizon planning; long-term learning and adaptation; estimation in dynamic environments; fault tolerance and failure prediction; and online calibration. We hope that the workshop will foster opportunities for collaboration, with the goal of developing robust, integrated autonomous systems.



%======================================================================
\clearpage {\Large {\bf WS6. Aquatic Robotics: Ocean Science and Marine Systems}}\\[5mm]
%======================================================================

{\bf  Schedule and Location:}

Full-day workshop, Monday, June 27, 2011\\
%9:30 am - 6:00 pm\\
Location: GFS 105 \\[4mm]
%%Room M2 (2$^{nd}$ floor), School of Economics and Business Administration, Universidad de Zaragoza, Spain.\\[4mm]

{\bf  Organizers:}

Ryan N. Smith, Queensland University of Technology\\
Noel Du Toit, California Institute of Technology\\
Burton H. Jones, University of Southern California\\
Kanna Rajan, Monterey Bay Aquarium Research Institute\\[4mm]


{\bf Description: }

This full day workshop will bring together researchers from aquatic robotics, AI, sensor networking, communications, physical and biological oceanography, and marine microbiology with the intention of creating collaborative links between these communities through sharing recent results and discussing future research directions. The workshop features technical talks by experts in marine robotic systems (single and multi-agent), as well as experts in marine science and biology. The organizers will have multiple platforms deployed off the Los Angeles coast during the workshop (and the conference). A novel aspect of this workshop is the concluding panel discussion: workshop attendees will collaborate to design and implement a sampling strategy/mission plan for the deployed vehicles. The objective is long-term data gathering on effluent plume development and vertical mixing off the Los Angeles, CA coast. The mission plan and actual execution will be available to all conference attendees, in real-time, via an online mission planning tool.


%======================================================================
\clearpage {\Large {\bf WS7.  Guaranteeing Motion Safety for Robots}}\\[5mm]
%======================================================================

{\bf  Schedule and Location:}

Full-day workshop, Monday, June 27, 2011\\
%9:30 am - 6:00 pm\\
Location: GFS 109 \\[4mm]
%%Room M4 (1$^{st}$ floor), School of Economics and Business Administration, Universidad de Zaragoza, Spain.\\[4mm]

{\bf  Organizers:}

Thierry Fraichard, INRIA Grenoble Rhone-Alpes\\
Kostas Bekris, University of Nevada, Reno \\
Jur van den Berg, University of North Carolina, Chapel-Hill\\[4mm]

{\bf Description: }

In the near future, it is expected that robotic systems will share the human living and working spaces. While moving (especially at high speed), automated vehicles, mobile manipulators and humanoid robots can be potentially dangerous should a collision occur. It is therefore critical to assert and characterize their motion safety, i.e., their guaranteed ability to avoid collision. While there is a rich literature on collision avoidance schemes, motion safety in the real world for systems with interesting dynamics remains an open problem. The purpose of this workshop is to disseminate recent research advances in guaranteed motion safety for complex robotic systems operating in challenging situations. Topics include, but are not limited to: collision avoidance, multi-robot coordination, safe reactive (re)planning approaches, safe motion for robots with kinodynamic constraints, control theoretic and probabilistic approaches to motion safety, and deadlock and livelock avoidance.


%======================================================================
\clearpage {\Large {\bf WS8.  Mobile Manipulation - Learning to Manipulate }}\\[5mm]
%======================================================================

{\bf  Schedule and Location:}

Full-day workshop, Monday, June 27, 2011\\
%9:30 am - 6:00 pm\\	
Location: GFS 118 \\[4mm]
%Room M5 (1$^{st}$ floor), School of Economics and Business Administration, Universidad de Zaragoza, Spain.\\[4mm]

{\bf  Organizers:}

Oliver Brock, TU Berlin\\
Dov Katz, UMass Amherst\\
Ellen Klingbeil, Stanford University\\
Sarah Osentoski, Bosch Research\\
Radu Bogdan Rusu, Willow Garage\\[4mm]

{\bf Description: }

This workshop explores new approaches to mobile manipulation with an emphasis on the relationship between machine learning and successful interaction in human environments. Autonomous manipulation in human environments is challenging because of the associated high dimensional state space and its inherent uncertainties. It requires perceptual and manipulation skills which are robust against sparse, incomplete and noisy information. In such environments, leveraging past experience, oftentimes resulting from the robot's own interactions, promises an increased robustness and reliability. Therefore, these challenges naturally connect autonomous manipulation to machine learning. The proposed workshop will explore these issues and feature research that addresses the challenges of autonomous manipulation in this context. The workshop will feature invited talks on the topics of machine learning, manipulation and perception. This would be the seventh in a series of mobile manipulation-centered workshops at RSS.



%======================================================================
\clearpage {\Large {\bf WS9.  Tutorial on 3D Point Cloud Processing: Point Cloud Library}}\\[5mm]
%======================================================================

{\bf  Schedule and Location:}

Full-day workshop, Friday, July 1, 2011\\
%9:30am-13:00pm\\
Location: GFS 116 \\[4mm]
%%Room P3, Paraninfo, Universidad de Zaragoza, Spain.\\[4mm]

{\bf  Organizers:}

Radu Bogdan Rusu, Willow Garage\\
Bastian Steder, University of Freiburg\\
Nico Blodow, Technical University of Munich\\
Dirk Holz, University of Bonn\\[4mm]

{\bf Description: }

With the advent of new, low-cost hardware such as the Kinect and continued efforts in advanced open source 3D point cloud processing, 3D perception gains more importance in robotics and other fields. We offer a tutorial on point cloud processing using the emerging Point Cloud Library (PCL), which presents an advanced and extensive approach to the subject and provides an overview of existing systems applying these techniques. Our goal is to provide an excellent reference material for students and researchers interested in this subject and take our guests through a complete application demonstration (given live) that combines subjects such as filtering, feature estimation, segmentation, registration, object recognition and surface reconstruction. The tutorial will be held using Microsoft Kinect/PrimeSense sensors, so we encourage the audience to bring theirs so we can follow all the steps together.

%======================================================================
\clearpage {\Large {\bf WS10.   A Comparison of Reinforcement Learning and Optimal Control Methods for Real-World Robotic Tasks}}\\[5mm]
%======================================================================

{\bf  Schedule and Location:}

Half-day workshop, Friday July 1, 2011 (afternoon)\\
%9:30am-13:00pm\\
Location: GFS 107 \\[4mm]
%%Aula Magna, Paraninfo, Universidad de Zaragoza, Spain.\\[4mm]

{\bf  Organizers:}


Freek Stulp,  University of Southern California\\
Evangelos Theodorou, University of Southern California\\
 Stefan Schaal, University of Southern California\\[4mm]

{\bf Description: }

Planning and control of robotic systems is challenging because robotic domains typically involve high-dimensional state and action spaces, uncertainty in state and action, and a large task space which requires generalization across tasks. In this workshop, we consider state-of-the-art methods and algorithms from the domains of reinforcement learning and optimal control, and discuss how they address the challenges above. Do they scale to high-dimensional tasks? Are they robust in the face of uncertainty? Do their solutions generalize well across tasks? Do they do so both in theory and in practice, when implemented on a physical robot system? We use a set of concrete robot tasks (performing flight maneuvers, humanoid locomotion and object manipulation) as use cases to focus the discussion.





%======================================================================
\clearpage {\Large {\bf WS11. Integrated Planning and Control}}\\[5mm]
%======================================================================

{\bf  Schedule and Location:}

Half-day workshop, Friday, July 1, 2011 (morning)\\
%9:30am-13:00pm\\
Location: GFS 118\\[4mm]
%%Room P1, Paraninfo, Universidad de Zaragoza, Spain.\\[4mm]

{\bf  Organizers:}

Surya Singh, ACFR, University of Sydney\\
Russ Tedrake, Massachusetts Institute of Technology\\
Peter Corke, CyPhy Lab, Queensland University of Technology\\[4mm]

{\bf Description: }

Dynamic systems have to integrate both (1) the plan and (2) the regulation. Not only are "interesting" systems dynamically constrained, but they often have several solution pathways available. So it the task in not only how, but also when. 

Solutions range from decision-policy searches to optimal controls to robust hardware. The goal of this of workshop is to provide a forum for both the science and the systems: from algorithms to tools to make this work.


%======================================================================
\clearpage {\Large {\bf WS12.  Human-Robot Interaction: Perspectives and Contributions to Robotics from the Human Sciences}}\\[5mm]
%======================================================================

{\bf  Schedule and Location:}

Full-day workshop, Friday, July 1, 2011\\
%9:30am-13:00pm\\
Location: GFS 101 \\[4mm]
%%Room P2, Paraninfo, Universidad de Zaragoza, Spain.\\[4mm]

{\bf  Organizers:}

Leila Takayama, Willow Garage\\
Maja Mataric, USC\\
Odest Chadwicke Jenkins, Brown\\
Holly Yanco, UMass Lowell\\
Brian Scassellati, Yale\\[4mm]

{\bf Description: }

As robotic capabilities improve, these robotic systems are becoming increasingly pervasive in real human-robot interaction settings, including (1) people interacting with autonomous robots, (2) people and robots sharing control, and (3) people tele-operating robots with direct control. The addition of humans into the equation presents a challenge and opportunity for engaging more seriously with the human sciences, including cognitive, behavioral, and social sciences. Without a deeper understanding of human user contexts, real user needs, user skills and limitations, and an ability to evaluate a robotic system’s performance in terms of user needs, robotics runs the risk of inventing technologies for the sake of the technologies themselves. We will address this risk by engaging in a discussion of what the human sciences have to offer robotics in terms of theoretical perspectives, empirical methods, useful concepts and models, and potential collaborations.

%======================================================================
\clearpage {\Large {\bf WS13.  Automated SLAM Evaluation}}\\[5mm]
%======================================================================

{\bf  Schedule and Location:}

Full-day workshop, Friday, July 1, 2011\\
%9:30am-13:00pm\\
Location: GFS 104 \\[4mm]
%%Room P2, Paraninfo, Universidad de Zaragoza, Spain.\\[4mm]

{\bf  Organizers:}

Michael Kaess, Massachusetts Institute of Technology\\
Giorgio Grisetti, Sapienza University of Rome / University of Freiburg\\
Kai Ni, Georgia Institute of Technology / Microsoft\\[4mm]


{\bf Description: }

Due to its high relevance, Simultaneous Localization and Mapping (SLAM) is one of the most deeply investigated fields in mobile robotics. Over the last two decades hundreds of publications addressing the topic have been produced, and effective approaches are nowadays available. However, having such a large choice can become an issue both for the SLAM user that does not know what to select and for the SLAM researcher that has to compare their method with hundreds of others. These problems could be solved by an automated and open SLAM evaluation system. However, the construction of such a system poses different conceptual and practical problems, such as: What are the properties of interest? How to measure them? And how to compare heterogeneous systems? In this workshop, we will discuss how to answer these questions and we will present a system to compare a well defined sub-problem in the domain of graph-based SLAM systems, the graph-optimization.



%======================================================================
\clearpage {\Large {\bf WS14.  Tutorial on Dynamic Vehicle Routing for Robotic Systems}}\\[5mm]
%======================================================================

{\bf  Schedule and Location:}

Full-day workshop, Friday, July 1, 2011\\
%9:30am-13:00pm\\
Location: GFS 105 \\[4mm]
%%Room P2, Paraninfo, Universidad de Zaragoza, Spain.\\[4mm]

{\bf  Organizers:}

Francesco Bullo, University of California, Santa Barbara\\
Emilio Frazzoli, Massachusetts Institute of Technology\\
Marco Pavone, NASA Jet Propulsion Laboratory, California Institute of Technology\\
Ketan Savla, Massachusetts Institute of Technology\\
Stephen L. Smith, University of Waterloo\\[4mm]


{\bf Description: }

This tutorial presents a joint algorithmic and queueing approach to the design of cooperative control and task allocation strategies for networks of robots, which must fulfill spatially-localized tasks generated over time by an exogenous process. As in queueing theory, task arrivals are modeled as a stochastic process, and queueing-style algorithms are required to enable robots to search, identify, allocate, prioritize, plan paths, and form teams. The design and analysis of these algorithms typically require a combination of receding-horizon resource allocation, distributed optimization, combinatorics and control. The key novelty lies in the integration of geometric and combinatorial aspects (e.g., coverage, traveling salesman problems) with stochastic and differential aspects (e.g., queueing effects and differential constraints on robot dynamics) in the context of distributed coordination of multi- robot networks. Applications of these algorithms are numerous, and include surveillance and monitoring missions, as well as transportation networks and automated material handling.	

%======================================================================
\clearpage {\Large {\bf WS15.  3D Exploration, Mapping, and Surveillance with Aerial Robots}}\\[5mm]
%======================================================================

{\bf  Schedule and Location:}

Full-day workshop, Friday, July 1, 2011\\
%9:30am-13:00pm\\
Location: GFS 108 \\[4mm]
%%Room P2, Paraninfo, Universidad de Zaragoza, Spain.\\[4mm]

{\bf  Organizers:}

Nathan Michael, University of Pennsylvania\\
Mac Schwager, University of Pennsylvania\\
 Vijay Kumar, University of Pennsylvania\\[4mm]

{\bf Description: }


Aerial robots play an increasingly important role in surveillance and exploration missions in both military and civilian domains, however most currently fielded aerial robots are teleoperated. Fully autonomous aerial robotic systems are within reach, though further research and development is needed. This workshop seeks to bring together theorists and experimentalists in aerial robotics to focus on the topics of exploration, mapping, and surveillance with aerial robots; including 3D SLAM, autonomous exploration, vehicle routing for surveillance, coverage control, and cooperative exploration. The workshop will emphasize the presentation of recent high-quality results in the pertinent areas and will encourage discussions to address the challenges of transitioning from theory to practice in this important and clearly motivated area of research. Participants will provide slides/videos for an on-line proceedings with the expectation of a journal-quality paper following the workshop. The organizers will pursue a special issue with IJRR or Autonomous Robots.



%======================================================================
\clearpage {\Large {\bf WS16.  Tutorial on Stochastic Models, Information Theory, and Lie Groups}}\\[5mm]
%======================================================================

{\bf  Schedule and Location:}

Half-day workshop, Friday, July 1, 2011 (afternoon)\\
%9:30am-13:00pm\\
Location: GFS 109 \\[4mm]
%%Room P2, Paraninfo, Universidad de Zaragoza, Spain.\\[4mm]

{\bf  Organizers:}

Gregory Chirikjian, Johns Hopkins University\\[4mm]

{\bf Description: }


In this tutorial the aim is to illustrate that concepts of information, communication, and motion come together in a natural way as stochastic processes on Lie groups. It will be shown how this formulation is natural to describe many problems in robot planning and localization including those associated with nonholonomic mobile robots and steering problems involving flexible needles for medical applications. It is well known that Robotics combines information processing and action in the physical world. Sensors collect information, multi-agent systems communicate, and individual robots execute actions accordingly. Both the sensing/communication and physical aspects of robotics are subject to noise. Hence stochastic models are appropriate, as is well-understood by those who work on filtering. The presentation of this half-day tutorial will be concrete, focusing on the Lie group of rigid-body motions/poses. And it will address issues such as: (1) How do the concepts of mean, covariance, convolution, and Fourier transform extend from Euclidean space to the case of pose data ? (2) How do information-theoretic inequalities generalize to non-Euclidean spaces ? (3) How can assembly planning and localization problems be formulated in a way that is not subject to artificial singularities introduced by Euler angles and other parameterizations ? (4) How are these applied in concrete applications in robotics and in structural biology ?

%======================================================================
\clearpage {\Large {\bf WS17.  HRI Workshop on Grounding Human-Robot Dialog for Spatial Tasks}}\\[5mm]
%======================================================================

{\bf  Schedule and Location:}

Half-day workshop, Friday, July 1, 2011 (morning)\\
%9:30am-13:00pm\\
Location: GFS 109 \\[4mm]
%%Room P2, Paraninfo, Universidad de Zaragoza, Spain.\\[4mm]

{\bf  Organizers:}

Thomas Kollar, Massachusetts Institute of Technology\\
Stefanie Tellex, Massachusetts Institute of Technology\\
Robert Ross, Dublin Institute of Technology\\
Antoine Raux, Honda Research Institute\\
Matthew Marge, Carnegie Mellon University\\[4mm]


{\bf Description: }

Unconstrained spoken language is an intuitive and flexible way for humans to command robots to perform spatial tasks such as navigation or manipulation. Dialog (i.e., conversational) interfaces offer a rich way to establish common ground between a person and a robot, and hence facilitate natural and flexible interaction. This workshop specifically aims to bring together communities studying robotics, natural language understanding, spatial cognition, and perception to build powerful spatial dialog systems for robots. We aim to bridge the gap between the theoretical understanding and practical application of spatial language understanding, create and discuss shared datasets and problems, and define key research problems and challenges.
