%%%%%%%%%%%%%%%%%%%%%%%%%%%%%%%%%%%%%%%%%%%%%%%%%%%%

\chapter*{Preface}

%%%%%%%%%%%%%%%%%%%%%%%%%%%%%%%%%%%%%%%%%%%%%%%%%%%%

                                                                     
                                             
\vspace{3cm}
\section*{Preface}
\begingroup{}
\Large
\vspace{1cm}

Welcome to RSS 2014 at UC Berkeley – the 10th incarnation of the RSS conference! 

\vspace{1mm}

RSS continues to attract some of the best work from across the whole spectrum of robotics research.  This year we have 57 papers being presented by authors from all around the world. Of the 57 papers, 6 will be presented through a long talk, and 51 will be presented through a short talk.  All 57 papers will in addition be presented through interactive sessions.   The RSS 2014 proceedings will be indexed by INSPEC and we are in the fortunate position to announce that all earlier RSS proceedings will also be indexed by INSPEC retroactively.  The proceedings themselves can be found at www.roboticsfoundation.org, the recently established RSS Foundation website, which also features the foundation’s goals, procedures and policies, and open letters to the RSS community.

\vspace{1mm}

In addition to the presentations of peer-reviewed papers, the program features 5 outstanding invited speakers: Nancy Amato, Genevieve Bell, Brad Nelson, Andrew Ng, and Chris Urmson; 2 early career spotlight presenters: Julie Shah and Ashutosh Saxena.  Also included are streaming of the world cup final on Sunday; an opening reception on Monday evening; a bay cruise conference banquet on Tuesday evening; lunchtime tours of the UC Berkeley robotics labs and an evening at Google on Wednesday.

\vspace{1mm}

This year’s RSS was made possible by many invisible hands that gave their support and put in long hours and hard work: the organizing committee, our sponsors, the area chairs, the many reviewers, the student volunteers, and so many more.  Special thanks to everybody who contributed!
Berkeley is well known for its wealth of cafes and restaurants.  We strongly encourage you to try to explore its variety of neighborhoods: Southside, Northside, Downtown, Ghourmet Ghetto, and Elmwood.

\vspace{1mm}

The RSS conference has seen solid growth in participation over the years, but the growth of the conference accelerated at a whole new level this year, with a good chance of doubling the number of conference participants.  We look forward to hosting all of you, and if we can help you with anything during the conference, please don’t hesitate to reach out to the local organizing committee—we wear the gold T-shirts!  

\vspace{1cm}

Pieter Abbeel, Sachin Patil, Lydia Kavraki, Dieter Fox
%
\endgroup{}
\normalsize

